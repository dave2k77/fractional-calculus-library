\section{Implementation Details}

The \hpfracc library is implemented in Python with comprehensive support for PyTorch, JAX, and NUMBA backends. The complete source code, documentation, and examples are publicly available at \url{https://github.com/dave2k77/fractional_calculus_library}.

\subsection{Spectral Autograd Framework}

The core innovation is the spectral autograd framework that enables gradient flow through fractional derivatives. The implementation uses FFT-based spectral methods with the following key components:

\begin{itemize}
    \item \textbf{Spectral Kernel Generation:} Complex-valued kernels $K_\alpha(\xi) = (i\xi)^\alpha$ computed using optimized FFT routines
    \item \textbf{Adjoint Operator:} Backward pass uses conjugate kernels $K_\alpha^*(\xi) = (-i\xi)^\alpha$ for proper gradient flow
    \item \textbf{Error Handling:} Multi-level fallback from PyTorch MKL FFT to NumPy and manual implementations
    \item \textbf{Learnable Parameters:} Bounded alpha parameterization with automatic gradient computation
\end{itemize}

\subsection{Fractional Derivative Algorithms}

The library implements three classical fractional derivative definitions:

\textbf{Riemann-Liouville:} Numerical integration using adaptive quadrature with convergence guarantees for $\alpha \in (0,2)$.

\textbf{Caputo:} L1 approximation with O(N log N) complexity and specialized boundary condition handling.

\textbf{Grünwald-Letnikov:} Discrete approximation with pre-computed binomial coefficients and adaptive step size control.

\subsection{Performance Optimizations}

\textbf{GPU Acceleration:} Automatic device detection, mixed-precision training, and cuDNN optimization for PyTorch backends.

\textbf{Memory Management:} Gradient checkpointing and memory-efficient algorithms for large-scale problems.

\textbf{Multi-Backend Support:} Unified API across PyTorch, JAX, and NUMBA with backend-specific optimizations.

\subsection{Validation and Testing}

The implementation includes comprehensive testing with 90% code coverage, validation against analytical solutions, and automated performance benchmarking. All numerical methods are validated against known analytical solutions for fractional relaxation equations and harmonic oscillators.

The complete implementation details, including full source code, extensive documentation, and interactive examples, are available in the public repository.
