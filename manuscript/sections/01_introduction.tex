\section{Introduction}

\subsection{Background and Motivation}

Fractional calculus has emerged as a powerful mathematical framework for modeling complex phenomena that exhibit memory effects, non-local behavior, and power-law dynamics. Unlike classical calculus, which deals with integer-order derivatives and integrals, fractional calculus extends these concepts to arbitrary real or complex orders, enabling the description of systems with long-range interactions, anomalous diffusion, and hereditary properties \cite{podlubny1999fractional, kilbas2006theory}.

The applications of fractional calculus span diverse scientific and engineering domains. In physics, fractional derivatives model anomalous transport in porous media \cite{metzler2000random}, viscoelastic behavior of materials \cite{mainardi2010fractional}, and quantum mechanical systems with memory effects \cite{laskin2000fractional}. In biology, fractional models describe cell growth dynamics \cite{west2003fractional}, neural signal propagation \cite{anastasio1994fractional}, and population dynamics with memory \cite{petras2011fractional}. Financial modeling benefits from fractional calculus through the description of long-memory processes in asset returns \cite{cont2001empirical} and option pricing with time-dependent volatility \cite{cartea2007fractional}.

However, solving fractional differential equations (FDEs) presents significant computational challenges. Traditional numerical methods often require fine temporal discretization to capture the non-local nature of fractional operators, leading to high computational costs and memory requirements. Analytical solutions exist only for a limited class of problems, leaving many real-world applications without tractable solutions.

The emergence of neural ordinary differential equations (Neural ODEs) \cite{chen2018neural} has revolutionized the field of differential equation solving by introducing learning-based approaches that can approximate complex dynamics without explicit knowledge of the underlying equations. This paradigm shift enables the solution of previously intractable problems through data-driven learning of the governing dynamics.

\subsection{Related Work}

Several frameworks have addressed aspects of fractional calculus and neural differential equations, but none provide the comprehensive integration offered by \hpfracc. Existing fractional calculus libraries include FracDiff \cite{fracdiff}, which focuses on financial time series analysis, and the Fractional Calculus Toolbox for MATLAB \cite{matlab_fractional}, which provides basic fractional operators but lacks machine learning integration.

Neural ODE implementations have proliferated since the seminal work of Chen et al. \cite{chen2018neural}, with frameworks like torchdiffeq \cite{torchdiffeq} and DiffEqFlux.jl \cite{diffeqflux} providing efficient ODE solvers with automatic differentiation. However, these frameworks lack support for fractional calculus and stochastic differential equations.

Stochastic differential equation solvers are available in specialized packages such as SDE.jl \cite{sde_jl} and PySDE \cite{pysde}, but they operate independently of neural network frameworks and lack the unified API that \hpfracc provides.

Physics-informed neural networks (PINNs) \cite{raissi2019physics} have demonstrated the power of incorporating physical constraints into neural network training, but existing implementations do not address the unique challenges of fractional differential equations.

\subsection{Contributions}

This work presents \hpfracc, the first comprehensive framework that unifies neural fractional ordinary differential equations with advanced stochastic differential equation solvers. Our primary contributions are:

\begin{enumerate}
    \item \textbf{Unified Neural Fractional ODE Framework}: A complete implementation of neural networks for fractional differential equations, extending the Neural ODE paradigm to fractional calculus with support for arbitrary fractional orders.
    
    \item \textbf{Production-Ready SDE Solvers}: Robust implementations of Euler-Maruyama, Milstein, and Heun methods for stochastic differential equations, integrated with neural network frameworks for learning-based SDE solving.
    
    \item \textbf{Modular Architecture}: A flexible, extensible design supporting multiple computation backends (PyTorch, JAX, NUMBA) with unified APIs for seamless integration across different mathematical domains.
    
    \item \textbf{Comprehensive Testing and Validation}: Extensive test suite with 85\%+ coverage, validation against analytical solutions, and performance benchmarking demonstrating practical utility.
    
    \item \textbf{Open-Source Accessibility}: Complete framework available as open-source software with PyPI package, comprehensive documentation, and extensive examples for immediate research use.
\end{enumerate}

The framework achieves 90\% implementation completion with robust error handling, parameter validation, and comprehensive testing. This work establishes a new standard for neural fractional calculus frameworks and opens new research directions in learning-based solution of complex differential equations.

\subsection{Paper Organization}

The remainder of this paper is organized as follows: Section 2 reviews the theoretical foundations of fractional calculus, neural ODEs, and stochastic differential equations. Section 3 provides a detailed treatment of fractional differential equations and their numerical solution. Section 4 describes the framework architecture and design principles. Section 5 details the implementation specifics and optimization strategies. Section 6 presents experimental results and performance analysis. Section 7 discusses limitations, future work, and research impact. Section 8 concludes with a summary of contributions and their significance.

\subsection{Software Availability}

\hpfracc is available as open-source software under the MIT license. The complete source code, documentation, and examples are hosted at \url{https://github.com/dave2k77/fractional_calculus_library}. The framework is distributed as a PyPI package (\texttt{hpfracc}) for easy installation and integration into existing Python workflows. Comprehensive documentation, including tutorials and API references, is available at \url{https://fractional-calculus-library.readthedocs.io/}.
