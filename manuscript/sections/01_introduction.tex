\section{Introduction}

\subsection{Core Scientific Problem and Hypothesis}

The fundamental challenge in computational fractional calculus is the **non-local nature of fractional operators**, which creates a fundamental incompatibility with standard automatic differentiation frameworks used in modern machine learning \citep{Gong2015ComputationalChallengeFDE, baydin2018automatic}. Unlike classical derivatives that depend only on local neighbourhoods, fractional derivatives require the entire function history, making traditional backpropagation techniques inapplicable.

**Our core hypothesis** is that spectral domain transformations can resolve this incompatibility by converting non-local fractional operations into local operations in the frequency domain, enabling the first practical implementation of automatic differentiation for fractional operators in neural networks.

This work presents \hpfracc (High-Performance Fractional Calculus Library), the first framework to successfully implement automatic differentiation for fractional operators through novel spectral autograd techniques, enabling neural networks to learn from systems with memory effects, power-law dynamics, and long-range correlations.

\subsection{Background and Motivation}

Fractional calculus has emerged as a powerful mathematical framework for modelling complex phenomena that exhibit memory effects, non-local behaviour, and power-law dynamics. Unlike classical calculus, which deals with integer-order derivatives and integrals, fractional calculus extends these concepts to arbitrary real or complex orders, enabling the description of systems with long-range interactions, anomalous diffusion, and hereditary properties \citep{podlubny1999fractional, kilbas2006theory}.

The applications of fractional calculus span diverse scientific and engineering domains. In physics, fractional derivatives model anomalous transport in porous media \citep{metzler2000random}, viscoelastic behaviour of materials \citep{mainardi2010fractional}, and quantum mechanical systems with memory effects \citep{laskin2000fractional}. In biology, fractional models describe cell growth dynamics \citep{west2003fractional}, neural signal propagation \citep{anastasio1994fractional}, and population dynamics with memory \citep{petras2011fractional}. Financial modelling benefits from fractional calculus through the description of long-memory processes in asset returns \citep{cont2001empirical} and option pricing with time-dependent volatility \citep{cartea2007fractional}.

However, solving fractional differential equations (FDEs) presents significant computational challenges \citep{Gong2015ComputationalChallengeFDE}. Traditional numerical methods often require fine temporal discretisation to capture the non-local nature of fractional operators, leading to high computational costs and memory requirements. Analytical solutions exist only for a limited class of problems, leaving many real-world applications without tractable solutions.

The emergence of neural ordinary differential equations (Neural ODEs) \citep{chen2018neural} has revolutionised the field of differential equation solving by introducing learning-based approaches that can approximate complex dynamics without explicit knowledge of the underlying equations. This paradigm shift enables the solution of previously intractable problems through data-driven learning of the governing dynamics.

\subsection{Related Work}

Several frameworks have addressed aspects of fractional calculus and neural differential equations, but none provide the comprehensive integration offered by \hpfracc. Existing fractional calculus libraries include FracDiff \citep{fracdiff}, which focuses on financial time series analysis, and the Fractional Calculus Toolbox for MATLAB \citep{matlab_fractional}, which provides basic fractional operators but lacks machine learning integration.

Neural ODE implementations have proliferated since the seminal work of \citet{chen2018neural}, with frameworks like torchdiffeq \citep{torchdiffeq} and DiffEqFlux.jl \citep{diffeqflux} providing efficient ODE solvers with automatic differentiation. However, these frameworks lack support for fractional calculus and stochastic differential equations.

Stochastic differential equation solvers are available in specialised packages such as SDE.jl \citep{sde_jl} and PySDE \citep{pysde}, but they operate independently of neural network frameworks and lack the unified API that \hpfracc provides.

Physics-informed neural networks (PINNs) \citep{raissi2019physics} have demonstrated the power of incorporating physical constraints into neural network training, but existing implementations do not address the unique challenges of fractional differential equations.

\subsection{Key Scientific Contributions}

This work presents \hpfracc, the first framework to successfully implement automatic differentiation for fractional operators, enabling neural networks to learn from systems with memory effects and long-range correlations. Our primary contributions are:

\begin{enumerate}
    \item \textbf{Novel Spectral Autograd Framework}: The first practical implementation of automatic differentiation for fractional operators through spectral domain transformations, resolving the fundamental incompatibility between non-local fractional operations and standard backpropagation techniques.
    
    \item \textbf{Potential Biomedical Applications}: Framework capabilities suggest potential for EEG-based brain-computer interface applications through fractional neural networks that capture long-memory effects, though actual biomedical experiments remain for future work.
    
    \item \textbf{Theoretical Rigor with Practical Impact}: Comprehensive mathematical proofs for convergence guarantees, error bounds, and stability analysis, combined with production-ready implementation achieving 19.7x speedup in adjoint training over standard methods.
    
    \item \textbf{Unified Neural Fractional ODE Framework}: Complete implementation extending the Neural ODE paradigm to fractional calculus, enabling physics-informed neural networks for fractional differential equations with memory effects.
    
    \item \textbf{Production-Ready Multi-Backend Support}: Robust implementation supporting PyTorch, JAX, and NUMBA backends with comprehensive testing (85\%+ coverage) and extensive validation against analytical solutions.
\end{enumerate}

**Potential Clinical Impact**: The framework's capabilities suggest potential for advancing brain-computer interfaces and neural signal processing through fractional neural networks, though actual biomedical experiments remain for future validation.

**Computational Impact**: \hpfracc achieves 19.7x speedup in adjoint training over standard methods while providing the first comprehensive framework for neural fractional calculus, opening new research directions in learning-based solution of complex differential equations with memory effects.

\subsection{Paper Organization}

The remainder of this paper is organized as follows: Section 2 reviews the theoretical foundations of fractional calculus, neural ODEs, and stochastic differential equations. Section 3 provides a detailed treatment of fractional differential equations and their numerical solution. Section 4 describes the framework architecture and design principles. Section 5 details the implementation specifics and optimization strategies. Section 6 presents experimental results and performance analysis. Section 7 discusses limitations, future work, and research impact. Section 8 concludes with a summary of contributions and their significance.

\subsection{Software Availability}

\hpfracc is available as open-source software under the MIT license. The complete source code, documentation, and examples are hosted at \url{https://github.com/dave2k77/fractional_calculus_library}. The framework is distributed as a PyPI package (\texttt{hpfracc}) for easy installation and integration into existing Python workflows. Comprehensive documentation, including tutorials and API references, is available at \url{https://fractional-calculus-library.readthedocs.io/}.
