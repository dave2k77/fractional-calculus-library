\section{Installation and System Requirements}

% Global settings for code listings to prevent overflow in appendices
\lstset{
    basicstyle=\ttfamily\footnotesize,
    breaklines=true,
    breakatwhitespace=true,
    columns=fullflexible,
    keepspaces=true,
    showstringspaces=false,
    frame=lines,
    xleftmargin=2em,
    xrightmargin=0em
}

\subsection{System Requirements}

\subsubsection{Hardware Requirements}
The framework supports a wide range of hardware configurations:
\begin{itemize}
    \item \textbf{CPU}: Any modern x86-64 processor (Intel i5/AMD Ryzen 5 or better recommended)
    \item \textbf{RAM}: Minimum 8GB, 16GB or more recommended for large problems
    \item \textbf{GPU}: NVIDIA GPU with CUDA support (RTX 3050 or better recommended)
    \item \textbf{Storage}: 2GB free space for installation, additional space for models and data
\end{itemize}

\subsubsection{Software Requirements}

\paragraph{Operating Systems}
Supported operating systems include:
\begin{itemize}
    \item \textbf{Linux}: Ubuntu 18.04+, CentOS 7+, RHEL 7+
    \item \textbf{macOS}: macOS 10.14+ (Mojave or later)
    \item \textbf{Windows}: Windows 10+ with WSL2 or native Python installation
\end{itemize}

\paragraph{Python Environment}
Python requirements:
\begin{itemize}
    \item \textbf{Python Version}: Python 3.8, 3.9, 3.10, or 3.11
    \item \textbf{Package Manager}: pip 20.0+ or conda 4.8+
    \item \textbf{Virtual Environment}: Recommended for isolation
\end{itemize}

\subsection{Installation Methods}

\subsubsection{PyPI Installation (Recommended)}
The simplest installation method is through PyPI:

\begin{lstlisting}[language=bash, caption=PyPI Installation]
# Install the main package
pip install hpfracc

# Install with optional dependencies for GPU support
pip install hpfracc[gpu]

# Install with all optional dependencies
pip install hpfracc[all]
\end{lstlisting}

\subsubsection{Conda Installation}
For conda users, installation is available through conda-forge:

\begin{lstlisting}[language=bash, caption=Conda Installation]
# Add conda-forge channel
conda config --add channels conda-forge

# Install the package
conda install hpfracc

# Install with GPU support
conda install hpfracc pytorch cudatoolkit
\end{lstlisting}

\subsubsection{Source Installation}
For development or custom modifications:

\begin{lstlisting}[language=bash, caption=Source Installation]
# Clone the repository
git clone https://github.com/dave2k77/fractional_calculus_library.git
cd fractional_calculus_library

# Install in development mode
pip install -e .

# Install development dependencies
pip install -r requirements_dev.txt
\end{lstlisting}

\subsection{Dependency Management}

\subsubsection{Core Dependencies}
The framework has the following core dependencies:
\begin{itemize}
    \item \textbf{NumPy}: Numerical computing foundation
    \item \textbf{SciPy}: Scientific computing algorithms
    \item \textbf{Matplotlib}: Plotting and visualization
    \item \textbf{PyTorch}: Deep learning framework (optional for basic usage)
\end{itemize}

\subsubsection{Optional Dependencies}
Optional dependencies provide additional functionality:
\begin{itemize}
    \item \textbf{PyTorch}: Full neural ODE and GPU support
    \item \textbf{JAX}: Alternative computation backend
    \item \textbf{NUMBA}: Just-in-time compilation for performance
    \item \textbf{torchdiffeq}: Advanced ODE solvers
    \item \textbf{torch-geometric}: Graph neural network support
\end{itemize}

\subsection{GPU Setup}

\subsubsection{CUDA Installation}
For NVIDIA GPU support:

\begin{lstlisting}[language=bash, caption=CUDA Installation]
# Check CUDA version
nvidia-smi

# Install PyTorch with CUDA support
pip install torch torchvision torchaudio --index-url https://download.pytorch.org/whl/cu118

# Verify installation
python -c "import torch; print(torch.cuda.is_available())"
\end{lstlisting}

\subsubsection{GPU Memory Management}
Tips for managing GPU memory:
\begin{itemize}
    \item \textbf{Batch Size}: Adjust batch size based on available GPU memory
    \item \textbf{Gradient Checkpointing}: Enable for memory-efficient training
    \item \textbf{Mixed Precision}: Use mixed precision training when available
    \item \textbf{Memory Monitoring}: Monitor GPU memory usage during training
\end{itemize}

\subsection{Environment Setup}

\subsubsection{Virtual Environment Creation}
Creating isolated Python environments:

\begin{lstlisting}[language=bash, caption=Virtual Environment Setup]
# Using venv (Python 3.3+)
python -m venv hpfracc_env
source hpfracc_env/bin/activate  # Linux/macOS
hpfracc_env\Scripts\activate     # Windows

# Using conda
conda create -n hpfracc_env python=3.10
conda activate hpfracc_env
\end{lstlisting}

\subsubsection{Environment Variables}
Recommended environment variables:

\begin{lstlisting}[language=bash, caption=Environment Variables]
# Set PyTorch to use CUDA if available
export CUDA_VISIBLE_DEVICES=0

# Enable PyTorch memory optimisation
export PYTORCH_CUDA_ALLOC_CONF=max_split_size_mb:128

# Set JAX to use GPU
export JAX_PLATFORM_NAME=gpu
\end{lstlisting}

\subsection{Verification and Testing}

\subsubsection{Basic Verification}
Verify the installation:

\begin{lstlisting}[language=python, caption=Installation Verification]
import hpfracc
print(f"HPFRACC version: {hpfracc.__version__}")

# Test basic functionality
from hpfracc.core.derivatives import create_fractional_derivative
print("Basic functionality working")
\end{lstlisting}

\subsubsection{Running Tests}
Run the test suite:

\begin{lstlisting}[language=bash, caption=Running Tests]
# Install test dependencies
pip install pytest pytest-cov

# Run all tests
python -m pytest tests/ -v

# Run tests with coverage
python -m pytest tests/ --cov=hpfracc --cov-report=html
\end{lstlisting}

\subsection{Troubleshooting}

\subsubsection{Common Installation Issues}

\paragraph{Permission Errors}
Solution for permission issues:

\begin{lstlisting}[language=bash, caption=Permission Fix]
# Use user installation
pip install --user hpfracc

# Or use virtual environment
python -m venv hpfracc_env
source hpfracc_env/bin/activate
pip install hpfracc
\end{lstlisting}

\paragraph{Version Conflicts}
Resolving version conflicts:

\begin{lstlisting}[language=bash, caption=Version Conflict Resolution]
# Check installed versions
pip list | grep torch
pip list | grep numpy

# Upgrade conflicting packages
pip install --upgrade torch numpy scipy

# Or use specific versions
pip install torch==2.0.1 numpy==1.24.3
\end{lstlisting}

\paragraph{GPU Issues}
Troubleshooting GPU problems:

\begin{lstlisting}[language=python, caption=GPU Troubleshooting]
import torch

# Check CUDA availability
print(f"CUDA available: {torch.cuda.is_available()}")
print(f"CUDA version: {torch.version.cuda}")

# Check GPU device
if torch.cuda.is_available():
    print(f"GPU device: {torch.cuda.get_device_name(0)}")
    print(f"GPU memory: {torch.cuda.get_device_properties(0).total_memory / 1e9:.1f} GB")
\end{lstlisting}

\subsubsection{Performance Issues}

\paragraph{Slow Performance}
Improving performance:
\begin{itemize}
    \item \textbf{GPU Usage}: Ensure PyTorch is using GPU acceleration
    \item \textbf{Batch Processing}: Use appropriate batch sizes for your hardware
    \item \textbf{Memory Management}: Enable gradient checkpointing for large models
    \item \textbf{Backend Selection}: Choose the most appropriate backend for your use case
\end{itemize}

\paragraph{Memory Issues}
Managing memory usage:
\begin{itemize}
    \item \textbf{Reduce Batch Size}: Decrease batch size to fit in available memory
    \item \textbf{Gradient Checkpointing}: Enable for memory-efficient training
    \item \textbf{Model Complexity}: Reduce model complexity for limited memory
    \item \textbf{Data Types}: Use appropriate data types (float32 vs float64)
\end{itemize}

\subsection{Development Setup}

\subsubsection{Development Dependencies}
For contributing to the framework:

\begin{lstlisting}[language=bash, caption=Development Setup]
# Install development dependencies
pip install -r requirements_dev.txt

# Install pre-commit hooks
pre-commit install

# Setup development environment
pip install -e .
\end{lstlisting}

\subsubsection{Code Quality Tools}
Development tools include:
\begin{itemize}
    \item \textbf{Black}: Code formatting
    \item \textbf{Flake8}: Linting and style checking
    \item \textbf{MyPy}: Type checking
    \item \textbf{Pre-commit}: Git hooks for code quality
\end{itemize}

\subsection{Container Deployment}

\subsubsection{Docker Installation}
Using Docker for deployment:

\begin{lstlisting}[language=dockerfile, caption=Dockerfile Example]
FROM python:3.10-slim

# Install system dependencies
RUN apt-get update && apt-get install -y \
    build-essential \
    && rm -rf /var/lib/apt/lists/*

# Install Python dependencies
COPY requirements.txt .
RUN pip install -r requirements.txt

# Install HPFRACC
RUN pip install hpfracc

# Set working directory
WORKDIR /app

# Copy application code
COPY . .

# Run application
CMD ["python", "app.py"]
\end{lstlisting}

\subsubsection{Singularity Installation}
For HPC environments:

\begin{lstlisting}[language=bash, caption=Singularity Recipe]
Bootstrap: docker
From: python:3.10-slim

%post
    apt-get update && apt-get install -y build-essential
    pip install hpfracc[gpu]

%environment
    export PYTHONPATH=/usr/local/lib/python3.10/site-packages:$PYTHONPATH

%runscript
    python "$@"
\end{lstlisting}

This comprehensive installation guide ensures that users can successfully install and configure \hpfracc for their specific use case, whether for basic usage, development, or production deployment.