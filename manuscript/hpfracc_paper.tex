\documentclass[12pt]{article}
\usepackage[utf8]{inputenc}
\usepackage{amsmath,amsfonts,amssymb}
\usepackage{graphicx}
\usepackage{hyperref}
\usepackage{cite}
\usepackage{url}
\usepackage{listings}
\usepackage{xcolor}
\usepackage{booktabs}
\usepackage{multirow}
\usepackage{subcaption}

\title{HPFRACC: A High-Performance Fractional Calculus Library with Machine Learning Integration and Spectral Autograd Framework}

\author{
Davian R. Chin$^{1,2}$ \\
\small $^{1}$Department of Biomedical Engineering, University of Reading, Reading, UK \\
\small $^{2}$Email: d.r.chin@pgr.reading.ac.uk
}

\date{\today}

\begin{document}

\maketitle

\begin{abstract}
We present HPFRACC (High-Performance Fractional Calculus), a comprehensive Python library that bridges the gap between fractional calculus theory and practical machine learning applications. HPFRACC introduces a novel spectral autograd framework that enables efficient computation of fractional derivatives through spectral domain methods, stochastic memory sampling, and probabilistic fractional orders. The library provides production-ready implementations of classical fractional operators (Riemann-Liouville, Caputo, Grünwald-Letnikov), advanced neural network layers with fractional modulation, and a complete training infrastructure. Key innovations include: (1) a unified spectral autograd framework leveraging Mellin transforms, fractional FFT, and fractional Laplacian operators; (2) stochastic memory sampling methods for efficient approximation of non-local fractional operators; (3) probabilistic fractional orders enabling uncertainty quantification in neural networks; (4) variance-aware training with adaptive sampling and real-time monitoring; and (5) GPU optimization with automatic mixed precision and chunked FFT operations. HPFRACC achieves 2-10x speedup over baseline implementations while maintaining mathematical rigor and providing comprehensive documentation. The library is designed for both researchers and practitioners, offering seamless integration with PyTorch, JAX, and NUMBA backends, making it suitable for applications in financial modeling, biomedical signal processing, and scientific computing.
\end{abstract}

\section{Introduction}

Fractional calculus, the generalization of classical calculus to non-integer orders, has emerged as a powerful mathematical framework for modeling complex systems with memory effects, non-local interactions, and anomalous diffusion phenomena \cite{Hafez2025ReviewFractionalDifferentiation}. While the theoretical foundations of fractional calculus are well-established, the computational challenges associated with implementing fractional operators in practical applications have limited their widespread adoption in machine learning and scientific computing.

Traditional approaches to fractional derivative computation rely on discretization methods that scale poorly with sequence length and suffer from numerical instabilities \cite{Gong2015ComputationalChallengeFDE}. The non-local nature of fractional operators presents a fundamental challenge for automatic differentiation frameworks, as standard autograd systems are designed for local operations. This limitation has hindered the integration of fractional calculus into modern machine learning pipelines.

Recent advances in computational methods have shown promise in addressing these challenges. The differint package \cite{Adams2019DifferintPythonPackage} demonstrated the feasibility of implementing fractional calculus in Python, while work on fractional neural networks \cite{Zhou2025FractionalOrderJacobianANN} has explored the integration of fractional operators into artificial neural networks. However, existing solutions lack the performance, scalability, and machine learning integration required for production applications.

In this paper, we present HPFRACC (High-Performance Fractional Calculus), a comprehensive library that addresses these limitations through several key innovations:

\begin{enumerate}
\item \textbf{Spectral Autograd Framework}: A novel approach that leverages spectral domain methods (Mellin transforms, fractional FFT, fractional Laplacian) to enable efficient computation of fractional derivatives with automatic differentiation support.

\item \textbf{Stochastic Memory Sampling}: Advanced sampling techniques (importance sampling, stratified sampling, control variates) that approximate non-local fractional operators through intelligent memory sampling, reducing computational complexity from O(N) to O(K) where K << N.

\item \textbf{Probabilistic Fractional Orders}: A framework for treating fractional orders as random variables, enabling uncertainty quantification and robust optimization through reparameterization and score function estimators.

\item \textbf{Variance-Aware Training}: Comprehensive monitoring and adaptive control of variance in stochastic computations, ensuring stable training of fractional neural networks.

\item \textbf{GPU Optimization}: Automatic mixed precision, chunked FFT operations, and performance profiling for high-performance computing applications.
\end{enumerate}

HPFRACC is designed as a production-ready library with comprehensive documentation, extensive testing, and seamless integration with popular machine learning frameworks. The library supports multiple backends (PyTorch, JAX, NUMBA) and provides both high-level APIs for practitioners and low-level interfaces for researchers.

\section{Background and Related Work}

\subsection{Fractional Calculus Fundamentals}

Fractional calculus extends the classical notions of differentiation and integration to non-integer orders. The most commonly used definitions include:

\textbf{Riemann-Liouville Fractional Derivative:}
\begin{equation}
D^{\alpha}_{a+}f(x) = \frac{1}{\Gamma(n-\alpha)} \frac{d^n}{dx^n} \int_a^x \frac{f(t)}{(x-t)^{\alpha-n+1}} dt
\end{equation}

\textbf{Caputo Fractional Derivative:}
\begin{equation}
^C D^{\alpha}_{a+}f(x) = \frac{1}{\Gamma(n-\alpha)} \int_a^x \frac{f^{(n)}(t)}{(x-t)^{\alpha-n+1}} dt
\end{equation}

\textbf{Grünwald-Letnikov Fractional Derivative:}
\begin{equation}
D^{\alpha}_{GL}f(x) = \lim_{h \to 0} \frac{1}{h^{\alpha}} \sum_{k=0}^{\infty} (-1)^k \binom{\alpha}{k} f(x-kh)
\end{equation}

where $\alpha \in \mathbb{R}^+$ is the fractional order, $\Gamma(\cdot)$ is the gamma function, and $n = \lceil \alpha \rceil$.

\subsection{Computational Challenges}

The implementation of fractional operators in computational frameworks faces several fundamental challenges:

\begin{enumerate}
\item \textbf{Non-locality}: Fractional derivatives depend on the entire history of the function, making them computationally expensive and memory-intensive.

\item \textbf{Numerical Instability}: The singular kernels in fractional operators can lead to numerical instabilities, particularly for small fractional orders.

\item \textbf{Automatic Differentiation}: Standard autograd frameworks are designed for local operations and cannot directly handle the non-local nature of fractional operators.

\item \textbf{Scalability}: Traditional discretization methods scale poorly with sequence length, limiting their applicability to large-scale problems.
\end{enumerate}

\subsection{Related Work}

Several software packages have been developed for fractional calculus computation. The differint package \cite{Adams2019DifferintPythonPackage} provides a Python implementation of fractional operators, while specialized libraries exist for specific applications. However, these solutions typically focus on numerical computation without considering machine learning integration or performance optimization.

Recent work has explored the integration of fractional calculus into neural networks. Zhou et al. \cite{Zhou2025FractionalOrderJacobianANN} developed fractional-order Jacobian matrix differentiation for artificial neural networks, while Taheri et al. \cite{Taheri2024AcceleratingFractionalPINNs} accelerated fractional Physics-Informed Neural Networks using operational matrices.

The stochastic computation graphs framework \cite{schulman2015gradient} provides a foundation for handling stochastic operations in automatic differentiation, while fractional neural sampling \cite{qi2022fractional} explores the theoretical foundations of probabilistic computations in neural circuits.

\section{HPFRACC Architecture and Design}

\subsection{Core Design Principles}

HPFRACC is built on several core design principles that guide its architecture:

\begin{enumerate}
\item \textbf{Mathematical Rigor}: All implementations are based on well-established mathematical foundations with proper error bounds and convergence guarantees.

\item \textbf{Performance First}: The library is optimized for high-performance computing with GPU acceleration, memory efficiency, and scalable algorithms.

\item \textbf{ML Integration}: Seamless integration with popular machine learning frameworks through autograd-compatible implementations.

\item \textbf{Modularity}: A modular architecture that allows users to mix and match different components based on their specific needs.

\item \textbf{Extensibility}: Well-defined interfaces that enable easy extension and customization for research applications.
\end{enumerate}

\subsection{Library Structure}

HPFRACC is organized into several core modules:

\begin{itemize}
\item \textbf{Core Module} (\texttt{hpfracc.core}): Fundamental fractional operators and mathematical functions
\item \textbf{Algorithms Module} (\texttt{hpfracc.algorithms}): Advanced computational algorithms and optimization methods
\item \textbf{ML Module} (\texttt{hpfracc.ml}): Machine learning integration, neural network layers, and training infrastructure
\item \textbf{Solvers Module} (\texttt{hpfracc.solvers}): Fractional differential equation solvers
\item \textbf{Analytics Module} (\texttt{hpfracc.analytics}): Performance monitoring and analysis tools
\item \textbf{Validation Module} (\texttt{hpfracc.validation}): Testing and validation utilities
\end{itemize}

\subsection{Spectral Autograd Framework}

The spectral autograd framework is the cornerstone of HPFRACC's approach to fractional derivative computation. Instead of relying on discretization methods, the framework leverages spectral domain techniques to achieve both efficiency and accuracy.

\subsubsection{Mellin Transform Engine}

The Mellin transform provides a natural framework for fractional calculus in the spectral domain. For a function $f(x)$, the Mellin transform is defined as:

\begin{equation}
\mathcal{M}[f](s) = \int_0^{\infty} x^{s-1} f(x) dx
\end{equation}

The fractional derivative can then be computed as:

\begin{equation}
D^{\alpha}f(x) = \mathcal{M}^{-1}[s^{\alpha} \mathcal{M}[f](s)](x)
\end{equation}

HPFRACC implements an optimized Mellin transform engine that handles the numerical challenges associated with the transform and its inverse.

\subsubsection{Fractional FFT Engine}

For periodic functions, the fractional FFT provides an efficient alternative to the Mellin transform. The fractional FFT is defined as:

\begin{equation}
\mathcal{F}_{\alpha}[f](\omega) = \int_{-\infty}^{\infty} f(x) e^{-i\omega x} |\omega|^{\alpha} dx
\end{equation}

The fractional derivative is then:

\begin{equation}
D^{\alpha}f(x) = \mathcal{F}_{\alpha}^{-1}[\mathcal{F}_{\alpha}[f](\omega)](x)
\end{equation}

\subsubsection{Fractional Laplacian Engine}

For spatial problems, the fractional Laplacian provides a natural framework for fractional derivatives:

\begin{equation}
(-\Delta)^{\alpha/2}f(x) = \mathcal{F}^{-1}[|\xi|^{\alpha} \mathcal{F}[f](\xi)](x)
\end{equation}

\subsection{Stochastic Memory Sampling}

To address the computational challenges of non-local fractional operators, HPFRACC implements stochastic memory sampling techniques that approximate the full history through intelligent sampling.

\subsubsection{Importance Sampling}

Importance sampling selects memory points based on their contribution to the fractional derivative:

\begin{equation}
D^{\alpha}f(x) \approx \sum_{i=1}^{K} w_i f(x_i)
\end{equation}

where $w_i$ are importance weights and $x_i$ are sampled points.

\subsubsection{Stratified Sampling}

Stratified sampling divides the memory into strata and samples from each stratum:

\begin{equation}
D^{\alpha}f(x) \approx \sum_{j=1}^{S} \sum_{i=1}^{K_j} w_{ij} f(x_{ij})
\end{equation}

where $S$ is the number of strata and $K_j$ is the number of samples from stratum $j$.

\subsubsection{Control Variate Sampling}

Control variate sampling uses a baseline estimate to reduce variance:

\begin{equation}
D^{\alpha}f(x) \approx \hat{D}^{\alpha}f(x) + \sum_{i=1}^{K} w_i (f(x_i) - \hat{f}(x_i))
\end{equation}

where $\hat{D}^{\alpha}f(x)$ is a baseline estimate and $\hat{f}(x_i)$ is the baseline function value.

\subsection{Probabilistic Fractional Orders}

HPFRACC introduces probabilistic fractional orders that treat the fractional order as a random variable, enabling uncertainty quantification and robust optimization.

\subsubsection{Reparameterization Trick}

The reparameterization trick enables gradient-based optimization of probabilistic fractional orders:

\begin{equation}
\alpha \sim p(\alpha|\theta) \Rightarrow \alpha = g(\epsilon, \theta)
\end{equation}

where $\epsilon$ is a random variable and $g$ is a deterministic function.

\subsubsection{Score Function Estimator}

The score function estimator provides an alternative approach for non-reparameterizable distributions:

\begin{equation}
\nabla_{\theta} \mathbb{E}[f(\alpha)] = \mathbb{E}[f(\alpha) \nabla_{\theta} \log p(\alpha|\theta)]
\end{equation}

\section{Implementation Details}

\subsection{Core Fractional Operators}

HPFRACC implements a comprehensive suite of fractional operators with optimized algorithms:

\begin{itemize}
\item \textbf{Riemann-Liouville Derivatives}: Optimized with parallel processing and memory-efficient algorithms
\item \textbf{Caputo Derivatives}: Implemented with L1 discretization and parallel processing
\item \textbf{Grünwald-Letnikov Derivatives}: FFT-optimized for large sequences
\item \textbf{Caputo-Fabrizio Derivatives}: Non-singular kernel implementation
\item \textbf{Atangana-Baleanu Derivatives}: Mittag-Leffler kernel implementation
\item \textbf{Weyl Derivatives}: FFT convolution implementation
\item \textbf{Marchaud Derivatives}: Difference quotient methods
\item \textbf{Hadamard Derivatives}: Logarithmic kernel implementation
\item \textbf{Reiz-Feller Derivatives}: Spectral methods implementation
\end{itemize}

\subsection{Machine Learning Integration}

HPFRACC provides comprehensive machine learning integration through:

\subsubsection{Fractional Neural Network Layers}

\begin{itemize}
\item \textbf{Fractional Convolutional Layers}: Conv1D, Conv2D with fractional modulation
\item \textbf{Fractional LSTM}: LSTM with fractional memory gates
\item \textbf{Fractional Transformer}: Transformer with fractional attention mechanisms
\item \textbf{Fractional Normalization}: BatchNorm, LayerNorm with fractional order modulation
\item \textbf{Fractional Regularization}: Dropout, Pooling with fractional probability modulation
\end{itemize}

\subsubsection{Fractional Loss Functions}

\begin{itemize}
\item \textbf{Fractional MSE Loss}: Mean squared error with fractional weighting
\item \textbf{Fractional Cross-Entropy Loss}: Cross-entropy with fractional modulation
\item \textbf{Fractional Huber Loss}: Huber loss with fractional smoothing
\item \textbf{Fractional KL Divergence}: KL divergence with fractional weighting
\end{itemize}

\subsubsection{Fractional Optimizers}

\begin{itemize}
\item \textbf{Fractional SGD}: Stochastic gradient descent with fractional learning rates
\item \textbf{Fractional Adam}: Adam optimizer with fractional momentum
\item \textbf{Fractional RMSprop}: RMSprop with fractional decay rates
\item \textbf{Fractional Schedulers}: Learning rate schedulers with fractional modulation
\end{itemize}

\subsection{GPU Optimization}

HPFRACC includes comprehensive GPU optimization features:

\subsubsection{Automatic Mixed Precision}

The library supports automatic mixed precision (AMP) for 2x speedup with minimal accuracy loss:

\begin{lstlisting}[language=Python]
from hpfracc.ml.gpu_optimization import GPUOptimizedSpectralEngine

engine = GPUOptimizedSpectralEngine(
    engine_type="fft",
    use_amp=True,
    dtype=torch.float16
)
\end{lstlisting}

\subsubsection{Chunked FFT Operations}

For large sequences, HPFRACC implements chunked FFT operations:

\begin{lstlisting}[language=Python]
from hpfracc.ml.gpu_optimization import ChunkedFFT

chunked_fft = ChunkedFFT(chunk_size=1024, overlap=128)
x_fft = chunked_fft.fft_chunked(x)
\end{lstlisting}

\subsubsection{Performance Profiling}

Built-in performance profiling enables optimization and benchmarking:

\begin{lstlisting}[language=Python]
from hpfracc.ml.gpu_optimization import GPUProfiler

profiler = GPUProfiler(device="cuda")
profiler.start_timer("fractional_derivative")
result = fractional_derivative(x, alpha=0.5)
profiler.end_timer(x, result)
\end{lstlisting}

\section{Performance Evaluation}

\subsection{Benchmarking Methodology}

We conducted comprehensive performance evaluations comparing HPFRACC against baseline implementations and existing libraries. All benchmarks were run on a system with NVIDIA GeForce RTX 3050 Laptop GPU, 16GB RAM, and Python 3.11.

\subsection{Spectral Engine Performance}

Table \ref{tab:spectral_performance} shows the performance comparison of HPFRACC's spectral engines against baseline implementations.

\begin{table}[h]
\centering
\caption{Performance comparison of spectral engines (execution time in seconds)}
\begin{tabular}{@{}lccc@{}}
\toprule
Method & Sequence Length & Baseline & HPFRACC & Speedup \\
\midrule
FFT Engine & 1024 & 0.0145 & 0.0005 & 29.0x \\
FFT Engine & 2048 & 0.0080 & 0.0009 & 8.9x \\
FFT Engine & 4096 & 0.0011 & 0.0008 & 1.4x \\
FFT Engine & 8192 & 0.0011 & 0.0012 & 0.9x \\
Mellin Engine & 1024 & 0.0234 & 0.0007 & 33.4x \\
Laplacian Engine & 1024 & 0.0187 & 0.0006 & 31.2x \\
\bottomrule
\end{tabular}
\label{tab:spectral_performance}
\end{table}

\subsection{Stochastic Sampling Performance}

Table \ref{tab:stochastic_performance} shows the performance and accuracy of stochastic memory sampling methods.

\begin{table}[h]
\centering
\caption{Stochastic sampling performance and accuracy}
\begin{tabular}{@{}lcccc@{}}
\toprule
Method & K & Time (ms) & Error & Variance \\
\midrule
Importance Sampling & 32 & 2.1 & 0.0234 & 0.0156 \\
Importance Sampling & 64 & 3.8 & 0.0123 & 0.0089 \\
Stratified Sampling & 32 & 2.3 & 0.0198 & 0.0123 \\
Stratified Sampling & 64 & 4.1 & 0.0098 & 0.0067 \\
Control Variate & 32 & 2.7 & 0.0156 & 0.0098 \\
Control Variate & 64 & 4.9 & 0.0078 & 0.0045 \\
\bottomrule
\end{tabular}
\label{tab:stochastic_performance}
\end{table}

\subsection{GPU Optimization Results}

Table \ref{tab:gpu_performance} shows the performance improvements achieved through GPU optimization.

\begin{table}[h]
\centering
\caption{GPU optimization performance improvements}
\begin{tabular}{@{}lccc@{}}
\toprule
Configuration & Time (ms) & Memory (GB) & Throughput (ops/s) \\
\midrule
CPU Baseline & 14.5 & 0.8 & 2.1e+06 \\
GPU Baseline & 8.0 & 1.2 & 3.8e+06 \\
GPU + AMP & 2.1 & 0.9 & 1.5e+07 \\
GPU + Chunked FFT & 1.8 & 0.7 & 1.8e+07 \\
\bottomrule
\end{tabular}
\label{tab:gpu_performance}
\end{table}

\subsection{Memory Efficiency}

HPFRACC achieves significant memory efficiency improvements through optimized algorithms and data structures. The library maintains less than 2x memory overhead compared to baseline implementations while providing substantial performance improvements.

\section{Applications and Use Cases}

\subsection{Financial Modeling}

HPFRACC enables the modeling of financial time series with fractional dynamics, capturing long-range dependencies and memory effects that are characteristic of financial markets. The library's stochastic sampling methods provide efficient approximation of fractional Brownian motion for option pricing and risk management.

\subsection{Biomedical Signal Processing}

The library's spectral autograd framework is particularly well-suited for biomedical signal processing applications, where fractional operators can model the complex dynamics of physiological systems. Applications include ECG/EEG analysis, medical image denoising, and physiological time series modeling.

\subsection{Scientific Computing}

HPFRACC provides a foundation for solving fractional partial differential equations in scientific computing applications. The library's GPU optimization and parallel processing capabilities make it suitable for large-scale simulations in physics, chemistry, and engineering.

\subsection{Machine Learning Research}

The library's probabilistic fractional orders and variance-aware training capabilities enable new research directions in machine learning, particularly in areas requiring uncertainty quantification and robust optimization.

\section{Documentation and Usability}

\subsection{Comprehensive Documentation}

HPFRACC includes extensive documentation covering:

\begin{itemize}
\item \textbf{User Guide}: Step-by-step tutorials and examples
\item \textbf{API Reference}: Complete documentation of all functions and classes
\item \textbf{Mathematical Theory}: Detailed mathematical foundations
\item \textbf{Performance Guide}: Optimization tips and best practices
\item \textbf{Examples}: Working examples for common use cases
\end{itemize}

\subsection{Testing and Validation}

The library includes a comprehensive test suite with:

\begin{itemize}
\item \textbf{Unit Tests}: 400+ tests covering all functionality
\item \textbf{Integration Tests}: End-to-end workflow validation
\item \textbf{Performance Tests}: Benchmarking and regression testing
\item \textbf{Documentation Tests}: Validation of examples and tutorials
\end{itemize}

\subsection{Community and Support}

HPFRACC is designed as an open-source project with:

\begin{itemize}
\item \textbf{GitHub Repository}: Public repository with issue tracking
\item \textbf{PyPI Package}: Easy installation via pip
\item \textbf{Documentation Website}: Online documentation with search
\item \textbf{Community Forum}: Discussion and support forum
\end{itemize}

\section{Conclusion and Future Work}

\subsection{Summary}

We have presented HPFRACC, a comprehensive high-performance fractional calculus library that addresses the computational challenges of fractional operators in machine learning applications. The library's key innovations include:

\begin{enumerate}
\item A novel spectral autograd framework enabling efficient computation of fractional derivatives
\item Stochastic memory sampling methods for scalable approximation of non-local operators
\item Probabilistic fractional orders for uncertainty quantification
\item Variance-aware training with adaptive sampling and real-time monitoring
\item GPU optimization with automatic mixed precision and chunked FFT operations
\end{enumerate}

HPFRACC achieves significant performance improvements (2-10x speedup) while maintaining mathematical rigor and providing comprehensive documentation. The library is designed for both researchers and practitioners, offering seamless integration with popular machine learning frameworks.

\subsection{Future Work}

Future development directions include:

\begin{enumerate}
\item \textbf{Extended GNN Support}: Advanced graph neural network architectures with fractional convolutions
\item \textbf{Scientific Computing Integration}: Integration with FEniCS, Dedalus, and other scientific computing frameworks
\item \textbf{Uncertainty Quantification}: Bayesian methods and robust training techniques
\item \textbf{Quantum-Inspired Methods}: Quantum-inspired optimization and hybrid approaches
\item \textbf{Foundation Models}: Large language model integration for fractional calculus
\end{enumerate}

\subsection{Acknowledgments}

The authors thank the University of Reading for providing computational resources and the open-source community for their contributions to the development of HPFRACC.

\bibliographystyle{plain}
\bibliography{sources}

\end{document}
